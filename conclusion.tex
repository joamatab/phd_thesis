\pagestyle{fancy}
\lhead{}
\renewcommand{\chaptermark}[1]{\markboth{\thechapter.\ #1}{}}
\chapter{Conclusions and future lines}
\label{ch:conclusions}

In this thesis we study different materials and structures for developing integrated all optical switches. Some of the achievements are:

\begin{itemize}
 \item A 150~ps response all optical switch with 10~dB extinction ratio~\cite{optoel}. The switch is based on a ring resonator whose resonances shift through free-carrier dispersion effect. The generated carriers have very short recombination times in comparison with other silicon waveguides.
 
 \item In order to enhance the Kerr effect, we use a silicon-nanocrystal-based slot waveguide, with which we demonstrate an ultrafast optical logic gate~\cite{Oton2010}.
 
 \item We perform a detailed characterization of the nonlinear dynamics of silicon-nanocrystal-based slot waveguides. We manage to distinguish all nonlinear processes and establish a quantitative comparison between those waveguides and standard Silicon strip waveguides. A higher nonlinear figure-of-merit is observed, together with very weak carrier effects~\cite{Matres2011,Matres:12}.
 
  \item Finally, we demonstrate that amorphous Silicon waveguides can show a figure of merit 7 times higher than regular SOI waveguides thanks to its higher band-gap energy, and therefore lower carrier generation. Moreover, its negligible carrier associated effects, makes amorphous silicon an ideal candidate for developing high speed all-optical switches~\cite{Matres2013}.
  
 \end{itemize}
 
 
The nonlinear figure of merit observed using amorphous silicon waveguides was higher than silicon-nanocrystal-based slot waveguides. This fact, together with lower loss and fewer fabrication steps makes amorphous silicon a more suitable candidate for all-optical switching and logic gating.
 

To enable mass manufacturing in standard CMOS foundries, all the materials considered are CMOS compatible. This will allow to produce high volumes of optical devices at low cost.

In conclusion, the good results obtained suggest that all optical switching will be achievable in the near future and with good prospects in terms of impact, not only in the field of research, but also from an industrial point of view.

It is also worth mentioning that it is possible to implement an all-optical switch beyond the use of elementary waveguides, using for example, structures such as semiconductor optical amplifiers or highly nonlinear fibers and glasses.

Apart from the main topic of the thesis, we also obtain very interesting experimental results (See Appendices~\ref{ch:experimentalSetups},~\ref{ch:paperBackscattering},~\ref{ch:paperPhase}):

\begin{itemize}

\item We develop an experimental technique for measuring backscattering in silicon microring resonators together with an analytical model that reproduces the experimental results and extracts the parameters of the rings from the resonances~\cite{Ballesteros2011}.
 
\item We present a modification of the ultrafast nonlinear setup, which is capable of measuring the phase response of an arbitrary photonic component. Examples of characterization of ring resonators and a corrugated waveguide are presented~\cite{Matres2013b}.
 
\end{itemize}


There are several future lines after this dissertation:

\begin{itemize}
 \item Demonstrate all-optical switching using amorphous silicon waveguides, after the promising results obtained in Ref.~\cite{Matres2013}.
 
 \item Reduce waveguide loss to increase the quality factor of ring resonators, as narrowing the resonances reduces the amount of phase shift necessary for switching and decreases its Energy per bit.
 
 \item Use a coupling-ratio-variable coupler, which is composed of a symmetrical Mach-Zehnder interferometer and thermo-optic phase shifters.~\cite{Kominato1993}.
 Having the ring in one of the arms of the MZI, we can control the coupling coefficient of the ring and ensure to work in the critical coupling condition, with maximum depth of the resonances, and improve the extinction ratio of the devices.
 
 \item Study the temperature drift of the ring resonances. One could compensate thermal variations with a heater or using a cladding material with opposite thermo-refractive coefficient than silicon~\cite{Teng2009,Zhou2009a,Han2007}.

\end{itemize}



% For the future, the advantage of all-optical switching is that scales with the bitrate, whereas power in conventional switches depends on the number of transitions. 
% For the future, the advantage of all-optical switching is that scales with the bitrate, whereas power in conventional switches depends on the number of transitions. 
% Therefore if we wish we reach higher and higher data speeds, one should consider the advantages of all-optical switching.


% However, what prevents us from using more power or presenting an Eye pattern is the effect of free-carriers generation, which alters the level of zero when  several pulses arrive.
% To solve this problem, several strategies to sweep carriers have been proposed, the most commonly used is to create a PN junction reverse polarized \cite{Turner-Foster2010}.
% We studied the use of different materials such as silicon nano-crystals~\cite{Matres2011,Matres2012} and amorphous silicon~\cite{Matres2013} to overcome the slow response of the free carriers.


% El capítulo de conclusiones tienes que reescribirlo. Tienes que extenderte un poco más, las conclusiones de una tesis tienen que ser una serie de puntos claramente indicados. Puedes llamar al capítulo "Conclusions and future prospects", y añadir un apartado con el trabajo futuro que tú crees que es necesiario para continuar.
% Yo usaría el formato de la tesis de Ana, que te mando en el email.
% All the materials considered were CMOS compatible and could be mass-produced using standard CMOS foundries, which is the major advantage of silicon photonic devices.
% For nonlinear measurements, we measured different nonlinear effects with a phase sensitive time resolved technique (appendix~\ref{ch:timeRes}).
% % strategies for a high speed and low cost solution for future optical interconnects. All optical switching will allow us to increase the speed using ultrafast nonlinear kerr effect and scale the power needed. Moreover all the materials considered are compatible with CMOS technology, which is crucial for large scale manufacturing at competitive prices.
% We have covered the building blocks and nonlinear effects to develop integrated all-optical switches.


% We conducted a comprehensive structural design work to optimize the nonlinear properties of different silicon guides and waveguides for all-optical switching.
% In particular, we used slotted waveguides, inserting a layer of silica with silicon nano-crystals in the middle of the guide.
% In the layer of low index for polarization perpendicular to this layer, the electric field increases due to the refractive index change, introducing a very significant nonlinear effect increase~\cite{Matres2011,Matres:12}.
% Such structures were first characterized in our group, obtaining very promising preliminary results in terms of powers of commutation and extinction factors~\cite{Martinez2010}.
% With this type of structures and conventional silicon guides manufactured in our facilities, we achieved very interesting results in switching~\cite{Oton} and logic gating~\cite{Oton2010}.

\pagestyle{plain}
\bibliographystyle{unsrt}
\bibliography{library}


% Also we characterized and simulated the dynamics of the carriers (appendix~\ref{ch:experimentalSetups}) and proprieties of ring resonators such as backscattering effects \cite{Ballesteros2011}.
% Finally we complemented ring resonator measurements with phase characterization to extract their parameters and group index in slow light corrugated waveguides ~\cite{Matres2013b}

\documentclass[oneside]{article}
\usepackage[utf8]{inputenc}
\usepackage{graphicx}
\usepackage{amsmath}
\title{Simulación Del Backscattering En Anillos Resonantes}
\author{Guillem B.G.}
%\setlength{\oddsidemargin}{0pt}
%\setlength{\evensidemargin}{0pt}

\begin{document}
\maketitle
\newpage
\section{Introducción}
El espectro en transmisión de los anillos observado experimentalmente a veces
muestra un dosdeblamiento de los picos que podemos asociar a un acoplamiento
entre los modos propagantes y contrapropagantes del anillo. Una posible causa
para que se produzca son las reflexiones que tienen lugar dentro del anillo
debidas a las irregularidades de las superficies como se explica en~\cite{surface-roughness}.
Además superpuesto siempre aparece un ruido de naturaleza estadística que es
el que se intentará estudiar. En~\cite{statistics} se mide experimentalmente que para una guía
recta, la distribución que controla la intensidad de la luz reflejada es una
exponencial negativa (i.e. $\lambda e^{-\lambda x}$). En~\cite{statistics} también se ofrece
una ecuación para evaluar el valor medio de la distribución en función de los
parámetros de la guía ($\alpha, L$)y una constante experimental ($H_0$) que 
depende tanto del modo como de la anchura de la guía.

\begin{equation}
   \langle R \rangle = H_0 (1-exp(-2\alpha L))/2\alpha
\end{equation}

\begin{figure}[!h]
    \centering
    \includegraphics[scale=0.5]{melloni.pdf}
    \caption{Gráfica extraida de~\cite{statistics}.}
\end{figure}

\section{Desarrollo}
En primer lugar se probó a darle una reflectividad al punto de scattering
distinta a cada longitud de onda según una distribución como la anterior
utilizando como longitud la longitud del anillo.

%%Gráficas con un par de fracasos
\begin{figure}[h]
    \centering
    \includegraphics[scale=0.3]{malr002.pdf}
    \caption{Resultados sin tener en cuenta $L_{eff}$. $R_{mean} = 0.002$}
    \label{mal1}
\end{figure}

\begin{figure}[h]
    \centering
    \includegraphics[scale=0.3]{malr02.pdf}
    \caption{Resultados sin tener en cuenta $L_{eff}$. $R_{mean} = 0.02$}
    \label{mal2}
\end{figure}

Pero como se puede observar los resultados no son nada satisfactorios porque
incluso aumentando mucho la reflectividad el desdoblamiento de los picos queda
enmascarado por el ruido debido a que la media de la exponencial decreciente
es igual a su desviación estándar.

Después de varios intentos se llegó a la conclusión de que el anillo, al  ser
una estructura resonante, muestra en realidad una longitud efectiva que podemos
relacionar con la primera derivada del desfase introducida por el mismo.
Podemos relacionar la expresión para el retardo de grupo [Heebner] con una
longitud efectiva:

\begin{equation}
   L_{eff}=L_{fisica} \Phi '
\end{equation}

Considerando esto último lo que se hace es:
\begin{enumerate}
\item Obtener la longitud efectiva del anillo para una $\lambda$ dada.
\item Con la ecuación (1) obtenemos la reflectividad media.
\item Obtenemos un valor aleatorio que siga la distribución densidad de
probabilidad con la media de 2).
\item Obtener la intensidad transmitida a esa longitud de onda con un punto de
reflectividad igual al obtenido en 3)
\item Repetir para la siguiente longitud de onda
\end{enumerate}

Siguiendo este procedimiento vemos que los picos vuelven a ser
discernibles. Esto es así porque el aumento de la desviación estándar solo
afecta mucho cuando hemos llegado justo a los puntos de la resonancia.

\section{Resultados}
Resultados obtenidos considerando la longitud efectiva del anillo, los parámetros
del acoplo guía anillo y la atenuación en dB aparece en las gráficas. El valor de
$H_0$ es 0.0128$cm^{-1}$

%%Graficas con los resultados guays
\begin{figure}[h]
    \centering
    \includegraphics[scale=0.5]{k3-a5.pdf}
    \caption{Resultado considerando $L_{eff}$}
    \label{r1g}
\end{figure}

\begin{figure}[!h]
    \centering
    \includegraphics[scale=0.5]{otro.pdf}
    \caption{Otro resultado considerando $L_{eff}$}
    \label{r2g}
\end{figure}

El uso de una longitud efectiva también concuerda con lo dicho en~\cite{scat-anillo},
donde se considera el indice de grupo y se observa como aumenta la reflexión
cuando este crece.
\newpage
\section{Mejoras}
La más importante es utilizar un retardo de grupo obtenido especificamente
para el anillo con acoplamiento entre sus dos modos, porque las resonancias
no caen en el mismo sitio que las del anillo normal debido al desdoblamiento,
y esto modificará el ruido estadístico que se observa en las simulaciones.
Otra posible mejora, que aun no se como implementar, sería considerar la
reflectividad del anillo como un parámetro distribuido en lugar de ocurrir
en un único punto.

 \begin{thebibliography}{99}
 \bibitem{surface-roughness}
 Surface-roughness-induced contradirectional coupling in ring and disk resonantors.
 Brent E. Little, Laine, Chu

 \bibitem{statistics}
 Statistics of backscattering in optical waveguides.
 Morichetti, Canciamilla, Melloni
 
 \bibitem{scat-anillo}
 Coherent Backscattering in optical microring resonators.
 Morichetti
 \end{thebibliography}

\end{document}
